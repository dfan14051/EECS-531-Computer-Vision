
% Default to the notebook output style

    


% Inherit from the specified cell style.




    
\documentclass[11pt]{article}

    
    
    \usepackage[T1]{fontenc}
    % Nicer default font (+ math font) than Computer Modern for most use cases
    \usepackage{mathpazo}

    % Basic figure setup, for now with no caption control since it's done
    % automatically by Pandoc (which extracts ![](path) syntax from Markdown).
    \usepackage{graphicx}
    % We will generate all images so they have a width \maxwidth. This means
    % that they will get their normal width if they fit onto the page, but
    % are scaled down if they would overflow the margins.
    \makeatletter
    \def\maxwidth{\ifdim\Gin@nat@width>\linewidth\linewidth
    \else\Gin@nat@width\fi}
    \makeatother
    \let\Oldincludegraphics\includegraphics
    % Set max figure width to be 80% of text width, for now hardcoded.
    \renewcommand{\includegraphics}[1]{\Oldincludegraphics[width=.8\maxwidth]{#1}}
    % Ensure that by default, figures have no caption (until we provide a
    % proper Figure object with a Caption API and a way to capture that
    % in the conversion process - todo).
    \usepackage{caption}
    \DeclareCaptionLabelFormat{nolabel}{}
    \captionsetup{labelformat=nolabel}

    \usepackage{adjustbox} % Used to constrain images to a maximum size 
    \usepackage{xcolor} % Allow colors to be defined
    \usepackage{enumerate} % Needed for markdown enumerations to work
    \usepackage{geometry} % Used to adjust the document margins
    \usepackage{amsmath} % Equations
    \usepackage{amssymb} % Equations
    \usepackage{textcomp} % defines textquotesingle
    % Hack from http://tex.stackexchange.com/a/47451/13684:
    \AtBeginDocument{%
        \def\PYZsq{\textquotesingle}% Upright quotes in Pygmentized code
    }
    \usepackage{upquote} % Upright quotes for verbatim code
    \usepackage{eurosym} % defines \euro
    \usepackage[mathletters]{ucs} % Extended unicode (utf-8) support
    \usepackage[utf8x]{inputenc} % Allow utf-8 characters in the tex document
    \usepackage{fancyvrb} % verbatim replacement that allows latex
    \usepackage{grffile} % extends the file name processing of package graphics 
                         % to support a larger range 
    % The hyperref package gives us a pdf with properly built
    % internal navigation ('pdf bookmarks' for the table of contents,
    % internal cross-reference links, web links for URLs, etc.)
    \usepackage{hyperref}
    \usepackage{longtable} % longtable support required by pandoc >1.10
    \usepackage{booktabs}  % table support for pandoc > 1.12.2
    \usepackage[inline]{enumitem} % IRkernel/repr support (it uses the enumerate* environment)
    \usepackage[normalem]{ulem} % ulem is needed to support strikethroughs (\sout)
                                % normalem makes italics be italics, not underlines
    

    
    
    % Colors for the hyperref package
    \definecolor{urlcolor}{rgb}{0,.145,.698}
    \definecolor{linkcolor}{rgb}{.71,0.21,0.01}
    \definecolor{citecolor}{rgb}{.12,.54,.11}

    % ANSI colors
    \definecolor{ansi-black}{HTML}{3E424D}
    \definecolor{ansi-black-intense}{HTML}{282C36}
    \definecolor{ansi-red}{HTML}{E75C58}
    \definecolor{ansi-red-intense}{HTML}{B22B31}
    \definecolor{ansi-green}{HTML}{00A250}
    \definecolor{ansi-green-intense}{HTML}{007427}
    \definecolor{ansi-yellow}{HTML}{DDB62B}
    \definecolor{ansi-yellow-intense}{HTML}{B27D12}
    \definecolor{ansi-blue}{HTML}{208FFB}
    \definecolor{ansi-blue-intense}{HTML}{0065CA}
    \definecolor{ansi-magenta}{HTML}{D160C4}
    \definecolor{ansi-magenta-intense}{HTML}{A03196}
    \definecolor{ansi-cyan}{HTML}{60C6C8}
    \definecolor{ansi-cyan-intense}{HTML}{258F8F}
    \definecolor{ansi-white}{HTML}{C5C1B4}
    \definecolor{ansi-white-intense}{HTML}{A1A6B2}

    % commands and environments needed by pandoc snippets
    % extracted from the output of `pandoc -s`
    \providecommand{\tightlist}{%
      \setlength{\itemsep}{0pt}\setlength{\parskip}{0pt}}
    \DefineVerbatimEnvironment{Highlighting}{Verbatim}{commandchars=\\\{\}}
    % Add ',fontsize=\small' for more characters per line
    \newenvironment{Shaded}{}{}
    \newcommand{\KeywordTok}[1]{\textcolor[rgb]{0.00,0.44,0.13}{\textbf{{#1}}}}
    \newcommand{\DataTypeTok}[1]{\textcolor[rgb]{0.56,0.13,0.00}{{#1}}}
    \newcommand{\DecValTok}[1]{\textcolor[rgb]{0.25,0.63,0.44}{{#1}}}
    \newcommand{\BaseNTok}[1]{\textcolor[rgb]{0.25,0.63,0.44}{{#1}}}
    \newcommand{\FloatTok}[1]{\textcolor[rgb]{0.25,0.63,0.44}{{#1}}}
    \newcommand{\CharTok}[1]{\textcolor[rgb]{0.25,0.44,0.63}{{#1}}}
    \newcommand{\StringTok}[1]{\textcolor[rgb]{0.25,0.44,0.63}{{#1}}}
    \newcommand{\CommentTok}[1]{\textcolor[rgb]{0.38,0.63,0.69}{\textit{{#1}}}}
    \newcommand{\OtherTok}[1]{\textcolor[rgb]{0.00,0.44,0.13}{{#1}}}
    \newcommand{\AlertTok}[1]{\textcolor[rgb]{1.00,0.00,0.00}{\textbf{{#1}}}}
    \newcommand{\FunctionTok}[1]{\textcolor[rgb]{0.02,0.16,0.49}{{#1}}}
    \newcommand{\RegionMarkerTok}[1]{{#1}}
    \newcommand{\ErrorTok}[1]{\textcolor[rgb]{1.00,0.00,0.00}{\textbf{{#1}}}}
    \newcommand{\NormalTok}[1]{{#1}}
    
    % Additional commands for more recent versions of Pandoc
    \newcommand{\ConstantTok}[1]{\textcolor[rgb]{0.53,0.00,0.00}{{#1}}}
    \newcommand{\SpecialCharTok}[1]{\textcolor[rgb]{0.25,0.44,0.63}{{#1}}}
    \newcommand{\VerbatimStringTok}[1]{\textcolor[rgb]{0.25,0.44,0.63}{{#1}}}
    \newcommand{\SpecialStringTok}[1]{\textcolor[rgb]{0.73,0.40,0.53}{{#1}}}
    \newcommand{\ImportTok}[1]{{#1}}
    \newcommand{\DocumentationTok}[1]{\textcolor[rgb]{0.73,0.13,0.13}{\textit{{#1}}}}
    \newcommand{\AnnotationTok}[1]{\textcolor[rgb]{0.38,0.63,0.69}{\textbf{\textit{{#1}}}}}
    \newcommand{\CommentVarTok}[1]{\textcolor[rgb]{0.38,0.63,0.69}{\textbf{\textit{{#1}}}}}
    \newcommand{\VariableTok}[1]{\textcolor[rgb]{0.10,0.09,0.49}{{#1}}}
    \newcommand{\ControlFlowTok}[1]{\textcolor[rgb]{0.00,0.44,0.13}{\textbf{{#1}}}}
    \newcommand{\OperatorTok}[1]{\textcolor[rgb]{0.40,0.40,0.40}{{#1}}}
    \newcommand{\BuiltInTok}[1]{{#1}}
    \newcommand{\ExtensionTok}[1]{{#1}}
    \newcommand{\PreprocessorTok}[1]{\textcolor[rgb]{0.74,0.48,0.00}{{#1}}}
    \newcommand{\AttributeTok}[1]{\textcolor[rgb]{0.49,0.56,0.16}{{#1}}}
    \newcommand{\InformationTok}[1]{\textcolor[rgb]{0.38,0.63,0.69}{\textbf{\textit{{#1}}}}}
    \newcommand{\WarningTok}[1]{\textcolor[rgb]{0.38,0.63,0.69}{\textbf{\textit{{#1}}}}}
    
    
    % Define a nice break command that doesn't care if a line doesn't already
    % exist.
    \def\br{\hspace*{\fill} \\* }
    % Math Jax compatability definitions
    \def\gt{>}
    \def\lt{<}
    % Document parameters
    \title{A5}
    
    
    

    % Pygments definitions
    
\makeatletter
\def\PY@reset{\let\PY@it=\relax \let\PY@bf=\relax%
    \let\PY@ul=\relax \let\PY@tc=\relax%
    \let\PY@bc=\relax \let\PY@ff=\relax}
\def\PY@tok#1{\csname PY@tok@#1\endcsname}
\def\PY@toks#1+{\ifx\relax#1\empty\else%
    \PY@tok{#1}\expandafter\PY@toks\fi}
\def\PY@do#1{\PY@bc{\PY@tc{\PY@ul{%
    \PY@it{\PY@bf{\PY@ff{#1}}}}}}}
\def\PY#1#2{\PY@reset\PY@toks#1+\relax+\PY@do{#2}}

\expandafter\def\csname PY@tok@w\endcsname{\def\PY@tc##1{\textcolor[rgb]{0.73,0.73,0.73}{##1}}}
\expandafter\def\csname PY@tok@c\endcsname{\let\PY@it=\textit\def\PY@tc##1{\textcolor[rgb]{0.25,0.50,0.50}{##1}}}
\expandafter\def\csname PY@tok@cp\endcsname{\def\PY@tc##1{\textcolor[rgb]{0.74,0.48,0.00}{##1}}}
\expandafter\def\csname PY@tok@k\endcsname{\let\PY@bf=\textbf\def\PY@tc##1{\textcolor[rgb]{0.00,0.50,0.00}{##1}}}
\expandafter\def\csname PY@tok@kp\endcsname{\def\PY@tc##1{\textcolor[rgb]{0.00,0.50,0.00}{##1}}}
\expandafter\def\csname PY@tok@kt\endcsname{\def\PY@tc##1{\textcolor[rgb]{0.69,0.00,0.25}{##1}}}
\expandafter\def\csname PY@tok@o\endcsname{\def\PY@tc##1{\textcolor[rgb]{0.40,0.40,0.40}{##1}}}
\expandafter\def\csname PY@tok@ow\endcsname{\let\PY@bf=\textbf\def\PY@tc##1{\textcolor[rgb]{0.67,0.13,1.00}{##1}}}
\expandafter\def\csname PY@tok@nb\endcsname{\def\PY@tc##1{\textcolor[rgb]{0.00,0.50,0.00}{##1}}}
\expandafter\def\csname PY@tok@nf\endcsname{\def\PY@tc##1{\textcolor[rgb]{0.00,0.00,1.00}{##1}}}
\expandafter\def\csname PY@tok@nc\endcsname{\let\PY@bf=\textbf\def\PY@tc##1{\textcolor[rgb]{0.00,0.00,1.00}{##1}}}
\expandafter\def\csname PY@tok@nn\endcsname{\let\PY@bf=\textbf\def\PY@tc##1{\textcolor[rgb]{0.00,0.00,1.00}{##1}}}
\expandafter\def\csname PY@tok@ne\endcsname{\let\PY@bf=\textbf\def\PY@tc##1{\textcolor[rgb]{0.82,0.25,0.23}{##1}}}
\expandafter\def\csname PY@tok@nv\endcsname{\def\PY@tc##1{\textcolor[rgb]{0.10,0.09,0.49}{##1}}}
\expandafter\def\csname PY@tok@no\endcsname{\def\PY@tc##1{\textcolor[rgb]{0.53,0.00,0.00}{##1}}}
\expandafter\def\csname PY@tok@nl\endcsname{\def\PY@tc##1{\textcolor[rgb]{0.63,0.63,0.00}{##1}}}
\expandafter\def\csname PY@tok@ni\endcsname{\let\PY@bf=\textbf\def\PY@tc##1{\textcolor[rgb]{0.60,0.60,0.60}{##1}}}
\expandafter\def\csname PY@tok@na\endcsname{\def\PY@tc##1{\textcolor[rgb]{0.49,0.56,0.16}{##1}}}
\expandafter\def\csname PY@tok@nt\endcsname{\let\PY@bf=\textbf\def\PY@tc##1{\textcolor[rgb]{0.00,0.50,0.00}{##1}}}
\expandafter\def\csname PY@tok@nd\endcsname{\def\PY@tc##1{\textcolor[rgb]{0.67,0.13,1.00}{##1}}}
\expandafter\def\csname PY@tok@s\endcsname{\def\PY@tc##1{\textcolor[rgb]{0.73,0.13,0.13}{##1}}}
\expandafter\def\csname PY@tok@sd\endcsname{\let\PY@it=\textit\def\PY@tc##1{\textcolor[rgb]{0.73,0.13,0.13}{##1}}}
\expandafter\def\csname PY@tok@si\endcsname{\let\PY@bf=\textbf\def\PY@tc##1{\textcolor[rgb]{0.73,0.40,0.53}{##1}}}
\expandafter\def\csname PY@tok@se\endcsname{\let\PY@bf=\textbf\def\PY@tc##1{\textcolor[rgb]{0.73,0.40,0.13}{##1}}}
\expandafter\def\csname PY@tok@sr\endcsname{\def\PY@tc##1{\textcolor[rgb]{0.73,0.40,0.53}{##1}}}
\expandafter\def\csname PY@tok@ss\endcsname{\def\PY@tc##1{\textcolor[rgb]{0.10,0.09,0.49}{##1}}}
\expandafter\def\csname PY@tok@sx\endcsname{\def\PY@tc##1{\textcolor[rgb]{0.00,0.50,0.00}{##1}}}
\expandafter\def\csname PY@tok@m\endcsname{\def\PY@tc##1{\textcolor[rgb]{0.40,0.40,0.40}{##1}}}
\expandafter\def\csname PY@tok@gh\endcsname{\let\PY@bf=\textbf\def\PY@tc##1{\textcolor[rgb]{0.00,0.00,0.50}{##1}}}
\expandafter\def\csname PY@tok@gu\endcsname{\let\PY@bf=\textbf\def\PY@tc##1{\textcolor[rgb]{0.50,0.00,0.50}{##1}}}
\expandafter\def\csname PY@tok@gd\endcsname{\def\PY@tc##1{\textcolor[rgb]{0.63,0.00,0.00}{##1}}}
\expandafter\def\csname PY@tok@gi\endcsname{\def\PY@tc##1{\textcolor[rgb]{0.00,0.63,0.00}{##1}}}
\expandafter\def\csname PY@tok@gr\endcsname{\def\PY@tc##1{\textcolor[rgb]{1.00,0.00,0.00}{##1}}}
\expandafter\def\csname PY@tok@ge\endcsname{\let\PY@it=\textit}
\expandafter\def\csname PY@tok@gs\endcsname{\let\PY@bf=\textbf}
\expandafter\def\csname PY@tok@gp\endcsname{\let\PY@bf=\textbf\def\PY@tc##1{\textcolor[rgb]{0.00,0.00,0.50}{##1}}}
\expandafter\def\csname PY@tok@go\endcsname{\def\PY@tc##1{\textcolor[rgb]{0.53,0.53,0.53}{##1}}}
\expandafter\def\csname PY@tok@gt\endcsname{\def\PY@tc##1{\textcolor[rgb]{0.00,0.27,0.87}{##1}}}
\expandafter\def\csname PY@tok@err\endcsname{\def\PY@bc##1{\setlength{\fboxsep}{0pt}\fcolorbox[rgb]{1.00,0.00,0.00}{1,1,1}{\strut ##1}}}
\expandafter\def\csname PY@tok@kc\endcsname{\let\PY@bf=\textbf\def\PY@tc##1{\textcolor[rgb]{0.00,0.50,0.00}{##1}}}
\expandafter\def\csname PY@tok@kd\endcsname{\let\PY@bf=\textbf\def\PY@tc##1{\textcolor[rgb]{0.00,0.50,0.00}{##1}}}
\expandafter\def\csname PY@tok@kn\endcsname{\let\PY@bf=\textbf\def\PY@tc##1{\textcolor[rgb]{0.00,0.50,0.00}{##1}}}
\expandafter\def\csname PY@tok@kr\endcsname{\let\PY@bf=\textbf\def\PY@tc##1{\textcolor[rgb]{0.00,0.50,0.00}{##1}}}
\expandafter\def\csname PY@tok@bp\endcsname{\def\PY@tc##1{\textcolor[rgb]{0.00,0.50,0.00}{##1}}}
\expandafter\def\csname PY@tok@fm\endcsname{\def\PY@tc##1{\textcolor[rgb]{0.00,0.00,1.00}{##1}}}
\expandafter\def\csname PY@tok@vc\endcsname{\def\PY@tc##1{\textcolor[rgb]{0.10,0.09,0.49}{##1}}}
\expandafter\def\csname PY@tok@vg\endcsname{\def\PY@tc##1{\textcolor[rgb]{0.10,0.09,0.49}{##1}}}
\expandafter\def\csname PY@tok@vi\endcsname{\def\PY@tc##1{\textcolor[rgb]{0.10,0.09,0.49}{##1}}}
\expandafter\def\csname PY@tok@vm\endcsname{\def\PY@tc##1{\textcolor[rgb]{0.10,0.09,0.49}{##1}}}
\expandafter\def\csname PY@tok@sa\endcsname{\def\PY@tc##1{\textcolor[rgb]{0.73,0.13,0.13}{##1}}}
\expandafter\def\csname PY@tok@sb\endcsname{\def\PY@tc##1{\textcolor[rgb]{0.73,0.13,0.13}{##1}}}
\expandafter\def\csname PY@tok@sc\endcsname{\def\PY@tc##1{\textcolor[rgb]{0.73,0.13,0.13}{##1}}}
\expandafter\def\csname PY@tok@dl\endcsname{\def\PY@tc##1{\textcolor[rgb]{0.73,0.13,0.13}{##1}}}
\expandafter\def\csname PY@tok@s2\endcsname{\def\PY@tc##1{\textcolor[rgb]{0.73,0.13,0.13}{##1}}}
\expandafter\def\csname PY@tok@sh\endcsname{\def\PY@tc##1{\textcolor[rgb]{0.73,0.13,0.13}{##1}}}
\expandafter\def\csname PY@tok@s1\endcsname{\def\PY@tc##1{\textcolor[rgb]{0.73,0.13,0.13}{##1}}}
\expandafter\def\csname PY@tok@mb\endcsname{\def\PY@tc##1{\textcolor[rgb]{0.40,0.40,0.40}{##1}}}
\expandafter\def\csname PY@tok@mf\endcsname{\def\PY@tc##1{\textcolor[rgb]{0.40,0.40,0.40}{##1}}}
\expandafter\def\csname PY@tok@mh\endcsname{\def\PY@tc##1{\textcolor[rgb]{0.40,0.40,0.40}{##1}}}
\expandafter\def\csname PY@tok@mi\endcsname{\def\PY@tc##1{\textcolor[rgb]{0.40,0.40,0.40}{##1}}}
\expandafter\def\csname PY@tok@il\endcsname{\def\PY@tc##1{\textcolor[rgb]{0.40,0.40,0.40}{##1}}}
\expandafter\def\csname PY@tok@mo\endcsname{\def\PY@tc##1{\textcolor[rgb]{0.40,0.40,0.40}{##1}}}
\expandafter\def\csname PY@tok@ch\endcsname{\let\PY@it=\textit\def\PY@tc##1{\textcolor[rgb]{0.25,0.50,0.50}{##1}}}
\expandafter\def\csname PY@tok@cm\endcsname{\let\PY@it=\textit\def\PY@tc##1{\textcolor[rgb]{0.25,0.50,0.50}{##1}}}
\expandafter\def\csname PY@tok@cpf\endcsname{\let\PY@it=\textit\def\PY@tc##1{\textcolor[rgb]{0.25,0.50,0.50}{##1}}}
\expandafter\def\csname PY@tok@c1\endcsname{\let\PY@it=\textit\def\PY@tc##1{\textcolor[rgb]{0.25,0.50,0.50}{##1}}}
\expandafter\def\csname PY@tok@cs\endcsname{\let\PY@it=\textit\def\PY@tc##1{\textcolor[rgb]{0.25,0.50,0.50}{##1}}}

\def\PYZbs{\char`\\}
\def\PYZus{\char`\_}
\def\PYZob{\char`\{}
\def\PYZcb{\char`\}}
\def\PYZca{\char`\^}
\def\PYZam{\char`\&}
\def\PYZlt{\char`\<}
\def\PYZgt{\char`\>}
\def\PYZsh{\char`\#}
\def\PYZpc{\char`\%}
\def\PYZdl{\char`\$}
\def\PYZhy{\char`\-}
\def\PYZsq{\char`\'}
\def\PYZdq{\char`\"}
\def\PYZti{\char`\~}
% for compatibility with earlier versions
\def\PYZat{@}
\def\PYZlb{[}
\def\PYZrb{]}
\makeatother


    % Exact colors from NB
    \definecolor{incolor}{rgb}{0.0, 0.0, 0.5}
    \definecolor{outcolor}{rgb}{0.545, 0.0, 0.0}



    
    % Prevent overflowing lines due to hard-to-break entities
    \sloppy 
    % Setup hyperref package
    \hypersetup{
      breaklinks=true,  % so long urls are correctly broken across lines
      colorlinks=true,
      urlcolor=urlcolor,
      linkcolor=linkcolor,
      citecolor=citecolor,
      }
    % Slightly bigger margins than the latex defaults
    
    \geometry{verbose,tmargin=1in,bmargin=1in,lmargin=1in,rmargin=1in}
    
    

    \begin{document}
    
    
    \maketitle
    
    

    
    \section{EECS 531: Computer Vision Assignment
5}\label{eecs-531-computer-vision-assignment-5}

\textbf{David Fan}

5/2/18

    In this notebook we will be translating the geometric computer vision
demo from MATLAB into Python and explaining what each section of the
code does with math and relevant background.

    \subsection{Prerequisites}\label{prerequisites}

These are different from the demo since the ones in the demo are about
setting up the Jupyter MATLAB kernel. Here we'll just be doing our
standard imports.

    \begin{Verbatim}[commandchars=\\\{\}]
{\color{incolor}In [{\color{incolor}21}]:} \PY{k+kn}{import} \PY{n+nn}{numpy} \PY{k}{as} \PY{n+nn}{np}
         \PY{k+kn}{import} \PY{n+nn}{matplotlib}\PY{n+nn}{.}\PY{n+nn}{pyplot} \PY{k}{as} \PY{n+nn}{plt}
         \PY{k+kn}{from} \PY{n+nn}{mpl\PYZus{}toolkits}\PY{n+nn}{.}\PY{n+nn}{mplot3d} \PY{k}{import} \PY{n}{Axes3D}
         \PY{k+kn}{import} \PY{n+nn}{mpl\PYZus{}toolkits}\PY{n+nn}{.}\PY{n+nn}{mplot3d} \PY{k}{as} \PY{n+nn}{a3}
         \PY{k+kn}{from} \PY{n+nn}{matplotlib} \PY{k}{import} \PY{n}{patches}
\end{Verbatim}


    \subsection{Build a Simple Virtual
World}\label{build-a-simple-virtual-world}

\subsubsection{Add something into the
world}\label{add-something-into-the-world}

    \begin{Verbatim}[commandchars=\\\{\}]
{\color{incolor}In [{\color{incolor}3}]:} \PY{k}{def} \PY{n+nf}{create\PYZus{}points}\PY{p}{(}\PY{p}{)}\PY{p}{:}
            \PY{p}{[}\PY{n}{Z}\PY{p}{,} \PY{n}{Y}\PY{p}{,} \PY{n}{X}\PY{p}{]} \PY{o}{=} \PY{n}{np}\PY{o}{.}\PY{n}{meshgrid}\PY{p}{(}\PY{p}{[}\PY{o}{\PYZhy{}}\PY{l+m+mf}{0.5}\PY{p}{,} \PY{l+m+mi}{0}\PY{p}{,} \PY{l+m+mf}{0.5}\PY{p}{]}\PY{p}{,} \PY{p}{[}\PY{o}{\PYZhy{}}\PY{l+m+mf}{0.5}\PY{p}{,} \PY{l+m+mi}{0}\PY{p}{,} \PY{l+m+mf}{0.5}\PY{p}{]}\PY{p}{,} \PY{p}{[}\PY{o}{\PYZhy{}}\PY{l+m+mf}{0.5}\PY{p}{,} \PY{l+m+mf}{0.}\PY{p}{,} \PY{l+m+mf}{0.5}\PY{p}{]}\PY{p}{,} \PY{n}{indexing} \PY{o}{=}\PY{l+s+s1}{\PYZsq{}}\PY{l+s+s1}{ij}\PY{l+s+s1}{\PYZsq{}}\PY{p}{)}
            
            \PY{c+c1}{\PYZsh{} This could easily be wrong. MATLAB matrix to Numpy array conversion can get hairy.}
            \PY{c+c1}{\PYZsh{} The line below attempts to match the matlab line: `points = [X(:), Y(:), Z(:)];`}
            \PY{c+c1}{\PYZsh{} The difference between row major and column major may cause issues down the line though.}
            \PY{n}{points} \PY{o}{=} \PY{n}{np}\PY{o}{.}\PY{n}{column\PYZus{}stack}\PY{p}{(}\PY{p}{(}\PY{n}{X}\PY{o}{.}\PY{n}{flatten}\PY{p}{(}\PY{n}{order}\PY{o}{=}\PY{l+s+s1}{\PYZsq{}}\PY{l+s+s1}{F}\PY{l+s+s1}{\PYZsq{}}\PY{p}{)}\PY{p}{,} \PY{n}{Y}\PY{o}{.}\PY{n}{flatten}\PY{p}{(}\PY{n}{order}\PY{o}{=}\PY{l+s+s1}{\PYZsq{}}\PY{l+s+s1}{F}\PY{l+s+s1}{\PYZsq{}}\PY{p}{)}\PY{p}{,} \PY{n}{Z}\PY{o}{.}\PY{n}{flatten}\PY{p}{(}\PY{n}{order}\PY{o}{=}\PY{l+s+s1}{\PYZsq{}}\PY{l+s+s1}{F}\PY{l+s+s1}{\PYZsq{}}\PY{p}{)}\PY{p}{)}\PY{p}{)}
            
            \PY{c+c1}{\PYZsh{} No clue how to replicate MATLAB\PYZsq{}s Jet function}
            \PY{n}{colors} \PY{o}{=} \PY{k+kc}{None}
            
            \PY{k}{return} \PY{p}{(}\PY{n}{points}\PY{p}{,} \PY{n}{colors}\PY{p}{)}
\end{Verbatim}


    This function is appropriately named. It returns a set of coordinates
for points to be plotted later on as well as a color value for each
point from the Jet colormap.

    \subsubsection{Plot the points}\label{plot-the-points}

    \begin{Verbatim}[commandchars=\\\{\}]
{\color{incolor}In [{\color{incolor}14}]:} \PY{k}{def} \PY{n+nf}{plot\PYZus{}points}\PY{p}{(}\PY{n}{points}\PY{p}{,} \PY{n}{colors}\PY{p}{,} \PY{n}{size} \PY{o}{=} \PY{l+m+mi}{50}\PY{p}{)}\PY{p}{:}
             \PY{n}{fig}\PY{p}{,} \PY{n}{ax} \PY{o}{=} \PY{n}{plt}\PY{o}{.}\PY{n}{subplots}\PY{p}{(}\PY{n}{projection}\PY{o}{=}\PY{l+s+s1}{\PYZsq{}}\PY{l+s+s1}{3d}\PY{l+s+s1}{\PYZsq{}}\PY{p}{)}
             \PY{n}{ax}\PY{o}{.}\PY{n}{scatter}\PY{p}{(}\PY{n}{points}\PY{p}{[}\PY{l+m+mi}{0}\PY{p}{]}\PY{p}{,} \PY{n}{points}\PY{p}{[}\PY{l+m+mi}{1}\PY{p}{]}\PY{p}{,} \PY{n}{points}\PY{p}{[}\PY{l+m+mi}{2}\PY{p}{]}\PY{p}{,} \PY{n}{colors}\PY{p}{,} \PY{n}{size}\PY{p}{)}
             \PY{k}{return} \PY{p}{(}\PY{n}{fig}\PY{p}{,} \PY{n}{ax}\PY{p}{)}
\end{Verbatim}


    This is relatively straight forward. It simply creates a 3D scatter plot
using a set of points and colors.

    \subsubsection{Set up a pair of cameras}\label{set-up-a-pair-of-cameras}

    \begin{Verbatim}[commandchars=\\\{\}]
{\color{incolor}In [{\color{incolor}5}]:} \PY{k}{def} \PY{n+nf}{preset\PYZus{}cameras}\PY{p}{(}\PY{p}{)}\PY{p}{:}
            \PY{n}{r} \PY{o}{=} \PY{l+m+mi}{5}
            \PY{n}{focal\PYZus{}length} \PY{o}{=} \PY{l+m+mf}{0.06}
            \PY{n}{width} \PY{o}{=} \PY{l+m+mi}{256}
            \PY{n}{height} \PY{o}{=} \PY{l+m+mi}{256}
            \PY{n}{film\PYZus{}width} \PY{o}{=} \PY{l+m+mf}{0.035}
            \PY{n}{film\PYZus{}height} \PY{o}{=} \PY{l+m+mf}{0.035}
            
            \PY{n}{alpha} \PY{o}{=} \PY{n}{np}\PY{o}{.}\PY{n}{pi}\PY{o}{/}\PY{l+m+mi}{6}
            \PY{n}{beta} \PY{o}{=} \PY{n}{np}\PY{o}{.}\PY{n}{pi}\PY{o}{/}\PY{l+m+mi}{6}
            \PY{n}{cam1} \PY{o}{=} \PY{p}{\PYZob{}}
                \PY{l+s+s1}{\PYZsq{}}\PY{l+s+s1}{position}\PY{l+s+s1}{\PYZsq{}}\PY{p}{:} \PY{p}{[}\PY{n}{r} \PY{o}{*} \PY{n}{np}\PY{o}{.}\PY{n}{cos}\PY{p}{(}\PY{n}{beta}\PY{p}{)} \PY{o}{*} \PY{n}{np}\PY{o}{.}\PY{n}{cos}\PY{p}{(}\PY{n}{alpha}\PY{p}{)} \PY{p}{,}  \PY{n}{r} \PY{o}{*} \PY{n}{np}\PY{o}{.}\PY{n}{cos}\PY{p}{(}\PY{n}{beta}\PY{p}{)} \PY{o}{*} \PY{n}{np}\PY{o}{.}\PY{n}{sin}\PY{p}{(}\PY{n}{alpha}\PY{p}{)}\PY{p}{,} \PY{n}{r} \PY{o}{*} \PY{n}{np}\PY{o}{.}\PY{n}{sin}\PY{p}{(}\PY{n}{beta}\PY{p}{)}\PY{p}{]}\PY{p}{,}
                \PY{l+s+s1}{\PYZsq{}}\PY{l+s+s1}{target}\PY{l+s+s1}{\PYZsq{}}\PY{p}{:} \PY{p}{[}\PY{l+m+mi}{0}\PY{p}{,} \PY{l+m+mi}{0}\PY{p}{,} \PY{l+m+mi}{0}\PY{p}{]}\PY{p}{,}
                \PY{l+s+s1}{\PYZsq{}}\PY{l+s+s1}{up}\PY{l+s+s1}{\PYZsq{}}\PY{p}{:} \PY{p}{[}\PY{l+m+mi}{0}\PY{p}{,} \PY{l+m+mi}{0}\PY{p}{,} \PY{l+m+mi}{1}\PY{p}{]}\PY{p}{,}
                \PY{l+s+s1}{\PYZsq{}}\PY{l+s+s1}{focal\PYZus{}length}\PY{l+s+s1}{\PYZsq{}}\PY{p}{:} \PY{n}{focal\PYZus{}length}\PY{p}{,}
                \PY{l+s+s1}{\PYZsq{}}\PY{l+s+s1}{film\PYZus{}width}\PY{l+s+s1}{\PYZsq{}}\PY{p}{:} \PY{n}{film\PYZus{}width}\PY{p}{,}
                \PY{l+s+s1}{\PYZsq{}}\PY{l+s+s1}{film\PYZus{}height}\PY{l+s+s1}{\PYZsq{}}\PY{p}{:} \PY{n}{film\PYZus{}height}\PY{p}{,}
                \PY{l+s+s1}{\PYZsq{}}\PY{l+s+s1}{width}\PY{l+s+s1}{\PYZsq{}}\PY{p}{:} \PY{n}{width}\PY{p}{,}
                \PY{l+s+s1}{\PYZsq{}}\PY{l+s+s1}{height}\PY{l+s+s1}{\PYZsq{}}\PY{p}{:} \PY{n}{height}
            \PY{p}{\PYZcb{}}
            
            \PY{n}{alpha} \PY{o}{=} \PY{n}{np}\PY{o}{.}\PY{n}{pi}\PY{o}{/}\PY{l+m+mi}{3}
            \PY{n}{beta} \PY{o}{=} \PY{n}{np}\PY{o}{.}\PY{n}{pi}\PY{o}{/}\PY{l+m+mi}{6}
            \PY{n}{cam2} \PY{o}{=} \PY{p}{\PYZob{}}
                \PY{l+s+s1}{\PYZsq{}}\PY{l+s+s1}{position}\PY{l+s+s1}{\PYZsq{}}\PY{p}{:} \PY{p}{[}\PY{n}{r} \PY{o}{*} \PY{n}{np}\PY{o}{.}\PY{n}{cos}\PY{p}{(}\PY{n}{beta}\PY{p}{)} \PY{o}{*} \PY{n}{np}\PY{o}{.}\PY{n}{cos}\PY{p}{(}\PY{n}{alpha}\PY{p}{)} \PY{p}{,}  \PY{n}{r} \PY{o}{*} \PY{n}{np}\PY{o}{.}\PY{n}{cos}\PY{p}{(}\PY{n}{beta}\PY{p}{)} \PY{o}{*} \PY{n}{np}\PY{o}{.}\PY{n}{sin}\PY{p}{(}\PY{n}{alpha}\PY{p}{)}\PY{p}{,} \PY{n}{r} \PY{o}{*} \PY{n}{np}\PY{o}{.}\PY{n}{sin}\PY{p}{(}\PY{n}{beta}\PY{p}{)}\PY{p}{]}\PY{p}{,}
                \PY{l+s+s1}{\PYZsq{}}\PY{l+s+s1}{target}\PY{l+s+s1}{\PYZsq{}}\PY{p}{:} \PY{p}{[}\PY{l+m+mi}{0}\PY{p}{,} \PY{l+m+mi}{0}\PY{p}{,} \PY{l+m+mi}{0}\PY{p}{]}\PY{p}{,}
                \PY{l+s+s1}{\PYZsq{}}\PY{l+s+s1}{up}\PY{l+s+s1}{\PYZsq{}}\PY{p}{:} \PY{p}{[}\PY{l+m+mi}{0}\PY{p}{,} \PY{l+m+mi}{0}\PY{p}{,} \PY{l+m+mi}{1}\PY{p}{]}\PY{p}{,}
                \PY{l+s+s1}{\PYZsq{}}\PY{l+s+s1}{focal\PYZus{}length}\PY{l+s+s1}{\PYZsq{}}\PY{p}{:} \PY{n}{focal\PYZus{}length}\PY{p}{,}
                \PY{l+s+s1}{\PYZsq{}}\PY{l+s+s1}{film\PYZus{}width}\PY{l+s+s1}{\PYZsq{}}\PY{p}{:} \PY{n}{film\PYZus{}width}\PY{p}{,}
                \PY{l+s+s1}{\PYZsq{}}\PY{l+s+s1}{film\PYZus{}height}\PY{l+s+s1}{\PYZsq{}}\PY{p}{:} \PY{n}{film\PYZus{}height}\PY{p}{,}
                \PY{l+s+s1}{\PYZsq{}}\PY{l+s+s1}{width}\PY{l+s+s1}{\PYZsq{}}\PY{p}{:} \PY{n}{width}\PY{p}{,}
                \PY{l+s+s1}{\PYZsq{}}\PY{l+s+s1}{height}\PY{l+s+s1}{\PYZsq{}}\PY{p}{:} \PY{n}{height}
            \PY{p}{\PYZcb{}}
            
            \PY{k}{return} \PY{p}{(}\PY{n}{cam1}\PY{p}{,} \PY{n}{cam2}\PY{p}{)}
\end{Verbatim}


    This function creates two dictionaries representing the two camera
objects. The dictionaries store the cameras' values. Specifically: - the
camera's position in space - the focal point (target) of the camera -
the up direction of the camera, the focal length of the camera - the
sensor height and width, and - the number of horizontal and vertical
pixels (width and height).

    \subsubsection{Plot camera}\label{plot-camera}

    \begin{Verbatim}[commandchars=\\\{\}]
{\color{incolor}In [{\color{incolor}6}]:} \PY{k}{def} \PY{n+nf}{camera\PYZus{}coordinate\PYZus{}system}\PY{p}{(}\PY{n}{cam}\PY{p}{)}\PY{p}{:}
            \PY{c+c1}{\PYZsh{} The axis of camera coordinate system}
            \PY{c+c1}{\PYZsh{} prinicipal axis}
            \PY{n}{zcam} \PY{o}{=} \PY{n}{cam}\PY{p}{[}\PY{l+s+s1}{\PYZsq{}}\PY{l+s+s1}{target}\PY{l+s+s1}{\PYZsq{}}\PY{p}{]} \PY{o}{\PYZhy{}} \PY{n}{cam}\PY{p}{[}\PY{l+s+s1}{\PYZsq{}}\PY{l+s+s1}{position}\PY{l+s+s1}{\PYZsq{}}\PY{p}{]}
            
            \PY{c+c1}{\PYZsh{} x axis should pend to principal axis and up direction}
            \PY{n}{xcam} \PY{o}{=} \PY{n}{np}\PY{o}{.}\PY{n}{cross}\PY{p}{(}\PY{n}{zcam}\PY{p}{,} \PY{n}{cam}\PY{p}{[}\PY{l+s+s1}{\PYZsq{}}\PY{l+s+s1}{up}\PY{l+s+s1}{\PYZsq{}}\PY{p}{]}\PY{p}{)}
            
            \PY{c+c1}{\PYZsh{} y axis should pend to principal axis and principal axis}
            \PY{n}{ycam} \PY{o}{=} \PY{n}{np}\PY{o}{.}\PY{n}{cross}\PY{p}{(}\PY{n}{zcam}\PY{p}{,} \PY{n}{xcam}\PY{p}{)}
            
            \PY{c+c1}{\PYZsh{} normalize to unit vector}
            \PY{n}{zcam} \PY{o}{=} \PY{n}{zcam} \PY{o}{/} \PY{n}{np}\PY{o}{.}\PY{n}{linalg}\PY{o}{.}\PY{n}{norm}\PY{p}{(}\PY{n}{zcam}\PY{p}{)}
            \PY{n}{xcam} \PY{o}{=} \PY{n}{xcam} \PY{o}{/} \PY{n}{np}\PY{o}{.}\PY{n}{linalg}\PY{o}{.}\PY{n}{norm}\PY{p}{(}\PY{n}{xcam}\PY{p}{)}
            \PY{n}{ycam} \PY{o}{=} \PY{n}{ycam} \PY{o}{/} \PY{n}{np}\PY{o}{.}\PY{n}{linalg}\PY{o}{.}\PY{n}{norm}\PY{p}{(}\PY{n}{ycam}\PY{p}{)}
            
            \PY{n}{origin} \PY{o}{=} \PY{n}{cam}\PY{p}{[}\PY{l+s+s1}{\PYZsq{}}\PY{l+s+s1}{position}\PY{l+s+s1}{\PYZsq{}}\PY{p}{]}
            
            \PY{k}{return} \PY{p}{(}\PY{n}{xcam}\PY{p}{,} \PY{n}{ycam}\PY{p}{,} \PY{n}{zcam}\PY{p}{,} \PY{n}{origin}\PY{p}{)}
\end{Verbatim}


    This function computes the coordinate system based on the input camera.
Z is along the principal axis, X is perpendicular to the principal axis
and runs up, and Y is perpendicular to both. It also returns the origin
value of the coordinate system.

    \begin{Verbatim}[commandchars=\\\{\}]
{\color{incolor}In [{\color{incolor}11}]:} \PY{k}{def} \PY{n+nf}{plot\PYZus{}camera}\PY{p}{(}\PY{n}{fig}\PY{p}{,} \PY{n}{ax}\PY{p}{,} \PY{n}{cam}\PY{p}{,} \PY{n}{label} \PY{o}{=} \PY{l+s+s1}{\PYZsq{}}\PY{l+s+s1}{\PYZsq{}}\PY{p}{,} \PY{n}{color}\PY{o}{=}\PY{p}{[}\PY{l+m+mf}{0.75}\PY{p}{,} \PY{l+m+mf}{0.75}\PY{p}{,} \PY{l+m+mf}{0.75}\PY{p}{]}\PY{p}{)}\PY{p}{:}
             \PY{k}{if} \PY{n}{label}\PY{p}{:}
                 \PY{n}{ax}\PY{o}{.}\PY{n}{text}\PY{p}{(}\PY{n}{cam}\PY{p}{[}\PY{l+s+s1}{\PYZsq{}}\PY{l+s+s1}{position}\PY{l+s+s1}{\PYZsq{}}\PY{p}{]}\PY{p}{[}\PY{l+m+mi}{0}\PY{p}{]}\PY{p}{,}\PY{n}{cam}\PY{p}{[}\PY{l+s+s1}{\PYZsq{}}\PY{l+s+s1}{position}\PY{l+s+s1}{\PYZsq{}}\PY{p}{]}\PY{p}{[}\PY{l+m+mi}{1}\PY{p}{]}\PY{p}{,}\PY{n}{cam}\PY{p}{[}\PY{l+s+s1}{\PYZsq{}}\PY{l+s+s1}{position}\PY{l+s+s1}{\PYZsq{}}\PY{p}{]}\PY{p}{[}\PY{l+m+mi}{2}\PY{p}{]}\PY{p}{,} \PY{n}{label}\PY{p}{)}
             
             \PY{c+c1}{\PYZsh{} Compute the camera coordinate system}
             \PY{n}{xcam}\PY{p}{,} \PY{n}{ycam}\PY{p}{,} \PY{n}{zcam}\PY{p}{,} \PY{n}{origin} \PY{o}{=} \PY{n}{camera\PYZus{}coordinate\PYZus{}system}\PY{p}{(}\PY{n}{cam}\PY{p}{)}
             
             \PY{c+c1}{\PYZsh{} The four corners of the rectangle}
             \PY{c+c1}{\PYZsh{} on the plane through focal points}
             \PY{n}{d} \PY{o}{=} \PY{n}{np}\PY{o}{.}\PY{n}{linalg}\PY{o}{.}\PY{n}{norm}\PY{p}{(}\PY{n}{cam}\PY{p}{[}\PY{l+s+s1}{\PYZsq{}}\PY{l+s+s1}{target}\PY{l+s+s1}{\PYZsq{}}\PY{p}{]} \PY{o}{\PYZhy{}} \PY{n}{cam}\PY{p}{[}\PY{l+s+s1}{\PYZsq{}}\PY{l+s+s1}{position}\PY{l+s+s1}{\PYZsq{}}\PY{p}{]}\PY{p}{)}
             \PY{n}{x} \PY{o}{=} \PY{l+m+mf}{0.5} \PY{o}{*} \PY{n}{cam}\PY{p}{[}\PY{l+s+s1}{\PYZsq{}}\PY{l+s+s1}{film\PYZus{}width}\PY{l+s+s1}{\PYZsq{}}\PY{p}{]} \PY{o}{*} \PY{n}{d} \PY{o}{/} \PY{n}{cam}\PY{p}{[}\PY{l+s+s1}{\PYZsq{}}\PY{l+s+s1}{focal\PYZus{}length}\PY{l+s+s1}{\PYZsq{}}\PY{p}{]}
             \PY{n}{y} \PY{o}{=} \PY{l+m+mf}{0.5} \PY{o}{*} \PY{n}{cam}\PY{p}{[}\PY{l+s+s1}{\PYZsq{}}\PY{l+s+s1}{film\PYZus{}height}\PY{l+s+s1}{\PYZsq{}}\PY{p}{]} \PY{o}{*} \PY{n}{d} \PY{o}{/} \PY{n}{cam}\PY{p}{[}\PY{l+s+s1}{\PYZsq{}}\PY{l+s+s1}{focal\PYZus{}length}\PY{l+s+s1}{\PYZsq{}}\PY{p}{]}
             
             \PY{n}{P1} \PY{o}{=} \PY{n}{origin} \PY{o}{+} \PY{n}{x} \PY{o}{*} \PY{n}{xcam} \PY{o}{+} \PY{n}{y} \PY{o}{*} \PY{n}{ycam} \PY{o}{+} \PY{n}{d} \PY{o}{*} \PY{n}{zcam}
             \PY{n}{P2} \PY{o}{=} \PY{n}{origin} \PY{o}{+} \PY{n}{x} \PY{o}{*} \PY{n}{xcam} \PY{o}{\PYZhy{}} \PY{n}{y} \PY{o}{*} \PY{n}{ycam} \PY{o}{+} \PY{n}{d} \PY{o}{*} \PY{n}{zcam}
             \PY{n}{P3} \PY{o}{=} \PY{n}{origin} \PY{o}{\PYZhy{}} \PY{n}{x} \PY{o}{*} \PY{n}{xcam} \PY{o}{\PYZhy{}} \PY{n}{y} \PY{o}{*} \PY{n}{ycam} \PY{o}{+} \PY{n}{d} \PY{o}{*} \PY{n}{zcam}
             \PY{n}{P4} \PY{o}{=} \PY{n}{origin} \PY{o}{\PYZhy{}} \PY{n}{x} \PY{o}{*} \PY{n}{xcam} \PY{o}{+} \PY{n}{y} \PY{o}{*} \PY{n}{ycam} \PY{o}{+} \PY{n}{d} \PY{o}{*} \PY{n}{zcam}
             
             \PY{c+c1}{\PYZsh{} Function to draw a line segment (p1, p2)}
             \PY{n}{connect} \PY{o}{=} \PY{k}{lambda} \PY{n}{p1}\PY{p}{,} \PY{n}{p2}\PY{p}{:} \PY{n}{ax}\PY{o}{.}\PY{n}{plot}\PY{p}{(}\PY{p}{[}\PY{n}{p1}\PY{p}{[}\PY{l+m+mi}{0}\PY{p}{]}\PY{p}{,} \PY{n}{p2}\PY{p}{[}\PY{l+m+mi}{0}\PY{p}{]}\PY{p}{]}\PY{p}{,} \PY{p}{[}\PY{n}{p1}\PY{p}{[}\PY{l+m+mi}{1}\PY{p}{]}\PY{p}{,} \PY{n}{p2}\PY{p}{[}\PY{l+m+mi}{2}\PY{p}{]}\PY{p}{]}\PY{p}{,} \PY{n}{color}\PY{o}{=}\PY{n}{color}\PY{p}{)}
             
             \PY{c+c1}{\PYZsh{} Plot line connect camera and target}
             \PY{n}{connect}\PY{p}{(}\PY{n}{cam}\PY{p}{[}\PY{l+s+s1}{\PYZsq{}}\PY{l+s+s1}{position}\PY{l+s+s1}{\PYZsq{}}\PY{p}{]}\PY{p}{,} \PY{n}{cam}\PY{p}{[}\PY{l+s+s1}{\PYZsq{}}\PY{l+s+s1}{target}\PY{l+s+s1}{\PYZsq{}}\PY{p}{]}\PY{p}{)}
             
             \PY{c+c1}{\PYZsh{} Plot line connect P1, P2, P3, P4}
             \PY{n}{Patch}\PY{p}{(}\PY{p}{)}\PY{o}{.}\PY{o}{.}\PY{o}{.}
\end{Verbatim}


    This plotting function plots a visual representation of the input camera
coordinate system. It plots the rectangle that the camera is viewing and
draws lines leading back to the origin.

Unfortunately, there's really no good Pythonic equivalent of the matlab
Patch function that does what the demo does so it seems my translation
efforts end here. Maybe if I had more time I could try to hack a
solution together, but unfortunately I don't have time for more than
just a direct translation so the notebook won't actually run all
together, but the idea should be there.

    \subsection{Show the Virtual World}\label{show-the-virtual-world}

    \begin{Verbatim}[commandchars=\\\{\}]
{\color{incolor}In [{\color{incolor}19}]:} \PY{n}{points}\PY{p}{,} \PY{n}{colors} \PY{o}{=} \PY{n}{create\PYZus{}points}\PY{p}{(}\PY{p}{)}
         \PY{n}{cam1}\PY{p}{,} \PY{n}{cam2} \PY{o}{=} \PY{n}{preset\PYZus{}cameras}\PY{p}{(}\PY{p}{)}
         \PY{c+c1}{\PYZsh{} print(\PYZsq{}points: \PYZbs{}n\PYZsq{}, points)}
         \PY{c+c1}{\PYZsh{} print(\PYZsq{}colors: \PYZbs{}n\PYZsq{}, colors)}
         \PY{n+nb}{print}\PY{p}{(}\PY{l+s+s1}{\PYZsq{}}\PY{l+s+s1}{camera 1: }\PY{l+s+se}{\PYZbs{}n}\PY{l+s+s1}{\PYZsq{}}\PY{p}{,} \PY{n}{cam1}\PY{p}{)}
         \PY{n+nb}{print}\PY{p}{(}\PY{l+s+s1}{\PYZsq{}}\PY{l+s+se}{\PYZbs{}n}\PY{l+s+s1}{camera 2: }\PY{l+s+se}{\PYZbs{}n}\PY{l+s+s1}{\PYZsq{}}\PY{p}{,} \PY{n}{cam2}\PY{p}{)}
\end{Verbatim}


    \begin{Verbatim}[commandchars=\\\{\}]
camera 1: 
 \{'position': [3.7500000000000004, 2.1650635094610964, 2.4999999999999996], 'target': [0, 0, 0], 'up': [0, 0, 1], 'focal\_length': 0.06, 'film\_width': 0.035, 'film\_height': 0.035, 'width': 256, 'height': 256\}

camera 2: 
 \{'position': [2.1650635094610973, 3.75, 2.4999999999999996], 'target': [0, 0, 0], 'up': [0, 0, 1], 'focal\_length': 0.06, 'film\_width': 0.035, 'film\_height': 0.035, 'width': 256, 'height': 256\}

    \end{Verbatim}

    Not much to say here. Just calling the methods from earlier.

    \begin{Verbatim}[commandchars=\\\{\}]
{\color{incolor}In [{\color{incolor} }]:} \PY{k}{def} \PY{n+nf}{lookthrough}\PY{p}{(}\PY{n}{cam}\PY{p}{)}\PY{p}{:}
            \PY{o}{.}\PY{o}{.}\PY{o}{.}
\end{Verbatim}


    Not actually sure what the MATLAB does here... It sets the axes to the
various camera axes.

    \begin{Verbatim}[commandchars=\\\{\}]
{\color{incolor}In [{\color{incolor} }]:} \PY{n}{fig}\PY{p}{,} \PY{n}{ax} \PY{o}{=} \PY{n}{plot\PYZus{}points}\PY{p}{(}\PY{n}{points}\PY{p}{,} \PY{n}{colors}\PY{p}{,} \PY{l+m+mi}{50}\PY{p}{)}
        \PY{n}{plot\PYZus{}camera}\PY{p}{(}\PY{n}{fig}\PY{p}{,} \PY{n}{ax}\PY{p}{,} \PY{n}{cam1}\PY{p}{,} \PY{l+s+s1}{\PYZsq{}}\PY{l+s+s1}{Cam1}\PY{l+s+s1}{\PYZsq{}}\PY{p}{,} \PY{p}{[}\PY{l+m+mi}{1}\PY{p}{,} \PY{l+m+mi}{0}\PY{p}{,} \PY{l+m+mi}{0}\PY{p}{]}\PY{p}{)}
        \PY{n}{plot\PYZus{}camera}\PY{p}{(}\PY{n}{fig}\PY{p}{,} \PY{n}{ax}\PY{p}{,} \PY{n}{cam2}\PY{p}{,} \PY{l+s+s1}{\PYZsq{}}\PY{l+s+s1}{Cam2}\PY{l+s+s1}{\PYZsq{}}\PY{p}{,} \PY{p}{[}\PY{l+m+mi}{0}\PY{p}{,} \PY{l+m+mi}{0}\PY{p}{,} \PY{l+m+mi}{1}\PY{p}{]}\PY{p}{)}
        \PY{n}{ax}\PY{o}{.}\PY{n}{set\PYZus{}title}\PY{p}{(}\PY{l+s+s2}{\PYZdq{}}\PY{l+s+s2}{The virtual world}\PY{l+s+s2}{\PYZdq{}}\PY{p}{)}
        
        \PY{n}{fig}\PY{p}{,} \PY{n}{ax} \PY{o}{=} \PY{n}{plot\PYZus{}points}\PY{p}{(}\PY{n}{points}\PY{p}{,} \PY{n}{colors}\PY{p}{,} \PY{l+m+mi}{50}\PY{p}{)}
        \PY{n}{plot\PYZus{}camera}\PY{p}{(}\PY{n}{fig}\PY{p}{,} \PY{n}{ax}\PY{p}{,} \PY{n}{cam1}\PY{p}{,} \PY{l+s+s1}{\PYZsq{}}\PY{l+s+s1}{\PYZsq{}}\PY{p}{,} \PY{p}{[}\PY{l+m+mi}{1}\PY{p}{,} \PY{l+m+mi}{0}\PY{p}{,} \PY{l+m+mi}{0}\PY{p}{]}\PY{p}{)}
        \PY{n}{lookthrough}\PY{p}{(}\PY{n}{cam1}\PY{p}{)}
        \PY{n}{ax}\PY{o}{.}\PY{n}{set\PYZus{}title}\PY{p}{(}\PY{l+s+s2}{\PYZdq{}}\PY{l+s+s2}{Look through camera 1}\PY{l+s+s2}{\PYZdq{}}\PY{p}{)}
        
        \PY{n}{fig}\PY{p}{,} \PY{n}{ax} \PY{o}{=} \PY{n}{plot\PYZus{}points}\PY{p}{(}\PY{n}{points}\PY{p}{,} \PY{n}{colors}\PY{p}{,} \PY{l+m+mi}{50}\PY{p}{)}
        \PY{n}{plot\PYZus{}camera}\PY{p}{(}\PY{n}{fig}\PY{p}{,} \PY{n}{ax}\PY{p}{,} \PY{n}{cam2}\PY{p}{,} \PY{l+s+s1}{\PYZsq{}}\PY{l+s+s1}{\PYZsq{}}\PY{p}{,} \PY{p}{[}\PY{l+m+mi}{0}\PY{p}{,} \PY{l+m+mi}{0}\PY{p}{,} \PY{l+m+mi}{1}\PY{p}{]}\PY{p}{)}
        \PY{n}{lookthrough}\PY{p}{(}\PY{n}{cam2}\PY{p}{)}
        \PY{n}{ax}\PY{o}{.}\PY{n}{set\PYZus{}title}\PY{p}{(}\PY{l+s+s2}{\PYZdq{}}\PY{l+s+s2}{Look through camera 2}\PY{l+s+s2}{\PYZdq{}}\PY{p}{)}
\end{Verbatim}


    Plots the virtual world and the view through each camera.

    \subsection{Camera Model}\label{camera-model}

\subsubsection{Euclidean transformation
matrix}\label{euclidean-transformation-matrix}

    \begin{Verbatim}[commandchars=\\\{\}]
{\color{incolor}In [{\color{incolor}23}]:} \PY{k}{def} \PY{n+nf}{ExtrinsicsMtx}\PY{p}{(}\PY{n}{cam}\PY{p}{)}\PY{p}{:}
             \PY{n}{xcam}\PY{p}{,} \PY{n}{ycam}\PY{p}{,} \PY{n}{zcam}\PY{p}{,} \PY{n}{origin} \PY{o}{=} \PY{n}{camera\PYZus{}coordinate\PYZus{}system}\PY{p}{(}\PY{n}{cam}\PY{p}{)}
             \PY{c+c1}{\PYZsh{} Rotation matrix}
             \PY{n}{R} \PY{o}{=} \PY{n}{np}\PY{o}{.}\PY{n}{stack}\PY{p}{(}\PY{p}{[}\PY{n}{xcam}\PY{p}{,} \PY{n}{ycam}\PY{p}{,} \PY{n}{zcam}\PY{p}{]}\PY{p}{)}
             \PY{n}{M} \PY{o}{=} \PY{p}{[}\PY{n}{R}\PY{p}{,} \PY{n}{np}\PY{o}{.}\PY{n}{dot}\PY{p}{(}\PY{o}{\PYZhy{}}\PY{n}{origin}\PY{p}{,} \PY{n}{R}\PY{p}{)}\PY{p}{]}
             \PY{k}{return} \PY{n}{M}
\end{Verbatim}


    The first step in transforming the coordinate system from Camera to
World is to perform a Euclidean transformation on the Camera coordinate
system.

\[
\begin{pmatrix}
X_{cam} \\
Y_{cam} \\
Z_{cam} \\
1
\end{pmatrix}
=
\begin{bmatrix}
R & \mathbf{t} \\
\mathbf{O}^T & 1
\end{bmatrix}
\begin{pmatrix}
X \\
Y \\
Z \\
1
\end{pmatrix}
\]

Here \(R\) is a \(3 \times 3\) rotation matrix and \(\mathbf t\) is a
\(3 \times 1\) translation vector.

    \subsubsection{Camera calibration
matrix}\label{camera-calibration-matrix}

    \begin{Verbatim}[commandchars=\\\{\}]
{\color{incolor}In [{\color{incolor}24}]:} \PY{k}{def} \PY{n+nf}{IntrinsicsMtx}\PY{p}{(}\PY{n}{cam}\PY{p}{)}\PY{p}{:}
             \PY{n}{cx} \PY{o}{=} \PY{p}{(}\PY{n}{cam}\PY{p}{[}\PY{l+s+s1}{\PYZsq{}}\PY{l+s+s1}{width}\PY{l+s+s1}{\PYZsq{}}\PY{p}{]} \PY{o}{+} \PY{l+m+mi}{1}\PY{p}{)} \PY{o}{*} \PY{o}{.}\PY{l+m+mi}{5}
             \PY{n}{cy} \PY{o}{=} \PY{p}{(}\PY{n}{cam}\PY{p}{[}\PY{l+s+s1}{\PYZsq{}}\PY{l+s+s1}{height}\PY{l+s+s1}{\PYZsq{}}\PY{p}{]} \PY{o}{+} \PY{l+m+mi}{1}\PY{p}{)} \PY{o}{*} \PY{o}{.}\PY{l+m+mi}{5}
             
             \PY{n}{fx} \PY{o}{=} \PY{n}{cam}\PY{p}{[}\PY{l+s+s1}{\PYZsq{}}\PY{l+s+s1}{focal\PYZus{}length}\PY{l+s+s1}{\PYZsq{}}\PY{p}{]} \PY{o}{*} \PY{n}{cam}\PY{p}{[}\PY{l+s+s1}{\PYZsq{}}\PY{l+s+s1}{width}\PY{l+s+s1}{\PYZsq{}}\PY{p}{]} \PY{o}{/} \PY{n}{cam}\PY{p}{[}\PY{l+s+s1}{\PYZsq{}}\PY{l+s+s1}{film\PYZus{}width}\PY{l+s+s1}{\PYZsq{}}\PY{p}{]}
             \PY{n}{fy} \PY{o}{=} \PY{n}{cam}\PY{p}{[}\PY{l+s+s1}{\PYZsq{}}\PY{l+s+s1}{focal\PYZus{}length}\PY{l+s+s1}{\PYZsq{}}\PY{p}{]} \PY{o}{*} \PY{n}{cam}\PY{p}{[}\PY{l+s+s1}{\PYZsq{}}\PY{l+s+s1}{height}\PY{l+s+s1}{\PYZsq{}}\PY{p}{]} \PY{o}{/} \PY{n}{cam}\PY{p}{[}\PY{l+s+s1}{\PYZsq{}}\PY{l+s+s1}{film\PYZus{}height}\PY{l+s+s1}{\PYZsq{}}\PY{p}{]}
             
             \PY{n}{K} \PY{o}{=} \PY{n}{np}\PY{o}{.}\PY{n}{asarray}\PY{p}{(}\PY{p}{[}\PY{p}{[}\PY{n}{fx}\PY{p}{,} \PY{l+m+mi}{0}\PY{p}{,} \PY{l+m+mi}{0}\PY{p}{]}\PY{p}{,}\PY{p}{[}\PY{l+m+mi}{0}\PY{p}{,} \PY{n}{fy}\PY{p}{,} \PY{l+m+mi}{0}\PY{p}{]}\PY{p}{,}\PY{p}{[}\PY{n}{cx}\PY{p}{,} \PY{n}{cy}\PY{p}{,} \PY{l+m+mi}{1}\PY{p}{]}\PY{p}{]}\PY{p}{)}
             \PY{k}{return} \PY{n}{K}
\end{Verbatim}


    The step above constructs the camera calibration matrix: \[
K = 
\begin{bmatrix}
\alpha_x & &x_0 \\
&\alpha_y & y_0 \\
&&1
\end{bmatrix}
\]

Where \(\alpha_x\) and \(alpha_y\) are the scaling parameters in the
image \(x\) and \(y\) directions and \((x_0, y_0)\) is the principal
point, the point where the optic axis intersects the image plane.

Of note is that the aspect ratio of the image is equal to:
\(\alpha_y/\alpha_x\)

The camera matrix is important because:

\[
\mathbf{x} =
\begin{pmatrix}
x\\y\\1
\end{pmatrix} = 
\frac{1}{f}
\begin{bmatrix}
\alpha_x & &x_0 \\
&\alpha_y & y_0 \\
&&1
\end{bmatrix}
\begin{pmatrix}
x_{cam}\\y_{cam}\\f
\end{pmatrix} =
K
\begin{pmatrix}
x_{cam}\\y_{cam}\\f
\end{pmatrix}
\]

    \subsubsection{Camera matrix}\label{camera-matrix}

    \begin{Verbatim}[commandchars=\\\{\}]
{\color{incolor}In [{\color{incolor}25}]:} \PY{k}{def} \PY{n+nf}{CameraMtx}\PY{p}{(}\PY{n}{cam}\PY{p}{)}\PY{p}{:}
             \PY{n}{M} \PY{o}{=} \PY{n}{ExtrinsicsMtx}\PY{p}{(}\PY{n}{cam}\PY{p}{)}
             \PY{n}{K} \PY{o}{=} \PY{n}{IntrinsicsMtx}\PY{p}{(}\PY{n}{cam}\PY{p}{)}
             \PY{n}{P} \PY{o}{=} \PY{n}{np}\PY{o}{.}\PY{n}{dot}\PY{p}{(}\PY{n}{M}\PY{p}{,} \PY{n}{K}\PY{p}{)}
             \PY{k}{return} \PY{n}{P}
\end{Verbatim}


    Here we calculate the Camera Matrix \(P\): \[
P = K
\begin{bmatrix}
1 & 0&0&0\\
0&1&0&0\\
0&0&1&0
\end{bmatrix}
M
\]

and \[
\mathbf x = P\mathbf X
\] Thus we have defined the \(3x4\) projection matrix from Euclidean
3-space to an image.

    \subsubsection{Generate the image pair}\label{generate-the-image-pair}

    \begin{Verbatim}[commandchars=\\\{\}]
{\color{incolor}In [{\color{incolor}26}]:} \PY{o}{.}\PY{o}{.}\PY{o}{.}
\end{Verbatim}


    \begin{Verbatim}[commandchars=\\\{\}]

          File "<ipython-input-26-82cca958dda9>", line 1
        def world2image(cam, points3d):
                                       \^{}
    SyntaxError: unexpected EOF while parsing


    \end{Verbatim}

    Here in the demo are a few functions that plot the projected points as
images using the above functions. As my translation no longer actually
works, there wasn't much point to translating this plotting code. If the
above ever gets fixed then the plotting code should go here.

    \subsection{Triangulation}\label{triangulation}

"Given the cameras and point pairs, reconstruct the 3D positions in
world coordinates of the point pairs."

\subsubsection{Linear triangulation
method}\label{linear-triangulation-method}

    \begin{Verbatim}[commandchars=\\\{\}]
{\color{incolor}In [{\color{incolor}27}]:} \PY{k}{def} \PY{n+nf}{triangulate}\PY{p}{(}\PY{n}{points1}\PY{p}{,} \PY{n}{points2}\PY{p}{,} \PY{n}{P1}\PY{p}{,} \PY{n}{P2}\PY{p}{)}\PY{p}{:}
             \PY{n}{num\PYZus{}points} \PY{o}{=} \PY{n}{points1}\PY{o}{.}\PY{n}{size}\PY{p}{[}\PY{l+m+mi}{0}\PY{p}{]}
             \PY{n}{points3d} \PY{o}{=} \PY{n}{np}\PY{o}{.}\PY{n}{zeros}\PY{p}{(}\PY{p}{(}\PY{n}{num\PYZus{}points}\PY{p}{,} \PY{l+m+mi}{3}\PY{p}{)}\PY{p}{)}
             
             \PY{c+c1}{\PYZsh{} iterate over point pairs}
             \PY{k}{for} \PY{n}{i} \PY{o+ow}{in} \PY{n+nb}{range}\PY{p}{(}\PY{l+m+mi}{1}\PY{p}{,} \PY{n}{num\PYZus{}points}\PY{p}{)}\PY{p}{:}
                 \PY{n}{points3d}\PY{p}{[}\PY{n}{i}\PY{p}{]} \PY{o}{=} \PY{n}{triangulationOnePoint}\PY{p}{(}\PY{n}{points1}\PY{p}{(}\PY{n}{i}\PY{p}{)}\PY{o}{.}\PY{n}{T}\PY{p}{,} \PY{n}{points2}\PY{p}{(}\PY{n}{i}\PY{p}{)}\PY{o}{.}\PY{n}{T}\PY{p}{,} \PY{n}{P1}\PY{o}{.}\PY{n}{T}\PY{p}{,} \PY{n}{P2}\PY{o}{.}\PY{n}{T}\PY{p}{)}
             \PY{k}{return} \PY{n}{points3d}
\end{Verbatim}


    This function is relatively simple as the next helper function is the
one that does the heavy lifting. This one just iterates over all point
pairs in the two sets of points and passes them to the helper function.
The return is used to set the 3D reconstruction coordinates.

The problem of triangulation can be formatted such that:

If we know: * \(P\) and \(P'\) * \(\mathbf x\) and \(\mathbf x'\) We
compute the reconstruction, \(\hat{x}\).

    \begin{Verbatim}[commandchars=\\\{\}]
{\color{incolor}In [{\color{incolor}30}]:} \PY{k}{def} \PY{n+nf}{triangulationOnePoint}\PY{p}{(}\PY{n}{point1}\PY{p}{,} \PY{n}{point2}\PY{p}{,} \PY{n}{P1}\PY{p}{,} \PY{n}{P2}\PY{p}{)}\PY{p}{:}
             \PY{c+c1}{\PYZsh{} Construct A}
             \PY{n}{A} \PY{o}{=} \PY{n}{np}\PY{o}{.}\PY{n}{zeros}\PY{p}{(}\PY{p}{(}\PY{l+m+mi}{4}\PY{p}{,}\PY{l+m+mi}{4}\PY{p}{)}\PY{p}{)}
             \PY{n}{A}\PY{p}{[}\PY{l+m+mi}{0}\PY{p}{:}\PY{l+m+mi}{2}\PY{p}{]} \PY{o}{=} \PY{n}{point1}\PY{o}{.}\PY{n}{dot}\PY{p}{(}\PY{n}{P1}\PY{p}{[}\PY{l+m+mi}{2}\PY{p}{]}\PY{p}{)} \PY{o}{\PYZhy{}} \PY{n}{P1}\PY{p}{[}\PY{l+m+mi}{0}\PY{p}{:}\PY{l+m+mi}{2}\PY{p}{]}
             \PY{n}{A}\PY{p}{[}\PY{l+m+mi}{2}\PY{p}{:}\PY{l+m+mi}{4}\PY{p}{]} \PY{o}{=} \PY{n}{point2}\PY{o}{.}\PY{n}{dot}\PY{p}{(}\PY{n}{P2}\PY{p}{[}\PY{l+m+mi}{2}\PY{p}{]}\PY{p}{)} \PY{o}{\PYZhy{}} \PY{n}{P2}\PY{p}{[}\PY{l+m+mi}{0}\PY{p}{:}\PY{l+m+mi}{2}\PY{p}{]}
             
             \PY{c+c1}{\PYZsh{} Solve the optimization problem: min\PYZus{}x ||Ax|| s.t. ||x||=1}
             \PY{n}{\PYZus{}}\PY{p}{,}\PY{n}{\PYZus{}}\PY{p}{,}\PY{n}{V} \PY{o}{=} \PY{n}{numpy}\PY{o}{.}\PY{n}{linalg}\PY{o}{.}\PY{n}{svd}\PY{p}{(}\PY{n}{A}\PY{p}{)}
             \PY{n}{X} \PY{o}{=} \PY{n}{V}\PY{p}{[}\PY{p}{:}\PY{p}{,} \PY{o}{\PYZhy{}}\PY{l+m+mi}{1}\PY{p}{]}
             \PY{n}{X} \PY{o}{=} \PY{n}{X}\PY{o}{/}\PY{n}{X}\PY{p}{[}\PY{o}{\PYZhy{}}\PY{l+m+mi}{1}\PY{p}{]}
             
             \PY{c+c1}{\PYZsh{} Homogenous \PYZhy{}\PYZgt{} Inhomogenous}
             \PY{n}{point3d} \PY{o}{=} \PY{n}{X}\PY{p}{[}\PY{l+m+mi}{0}\PY{p}{:}\PY{l+m+mi}{3}\PY{p}{]}
             \PY{k}{return} \PY{n}{point3d}
\end{Verbatim}


    Here we are calculating the actual reconstruction of each point for each
point pair. We do this by solving the system of linear equations:

\[
\begin{bmatrix}
x\mathbf p^{3T} - \mathbf p^{1T}\\
y\mathbf p^{3T} - \mathbf p^{2T}\\
x'\mathbf p'^{3T} - \mathbf p'^{1T}\\
x\mathbf p'^{3T} - \mathbf p'^{2T}
\end{bmatrix}
\mathbf x
=
0
\]

    \begin{Verbatim}[commandchars=\\\{\}]
{\color{incolor}In [{\color{incolor} }]:} \PY{o}{.}\PY{o}{.}\PY{o}{.}
\end{Verbatim}


    The rest is just plotting stuff to see the reconstruction!


    % Add a bibliography block to the postdoc
    
    
    
    \end{document}
